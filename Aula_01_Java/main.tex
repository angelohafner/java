\documentclass[aspectratio=169]{beamer}

\usetheme{Madrid}
\usecolortheme{default}

\graphicspath{{figuras/}}

\usepackage[T1]{fontenc}
\usepackage[utf8]{inputenc}
\usepackage[brazil]{babel}

\usepackage{minted}
\usepackage{graphicx}
\usepackage{xcolor}

\setminted{
	linenos,
	breaklines,
	fontsize=\normal,
	numbersep=6pt,
	xleftmargin=2.2em
}



\title{Introdução à Programação em Java}
\subtitle{Conceitos Básicos}
\author{Angelo Alfredo Hafner}
\date{\today}

\begin{document}

\begin{frame}
  \titlepage
\end{frame}

\section{Visão geral}

\begin{frame}{Onde o java é diferente?}
\centering
\includegraphics[width=15cm]{Fluxo_java.png}
\end{frame}

\begin{frame}{Onde Java é utilizado}
\centering
\includegraphics[height=7cm]{Utilizacao_java.png}
\end{frame}

\section{Ambiente de desenvolvimento}

\begin{frame}{O que precisa ser instalado}
\centering
\includegraphics[height=5cm]{O_que_instalar_java.png}
\end{frame}

\begin{frame}{Como funciona um programa Java}
\centering
\includegraphics[height=5cm]{Como_funciona_um_programa_java.png}
\end{frame}

\section{Estrutura básica}

\begin{frame}{Estrutura básica de um programa Java}
\centering
\includegraphics[height=7cm]{Estrutura_basica_de_um_programa_java.png}
\end{frame}

\section{Exercício}

\begin{frame}[fragile]{Exercício 01 — Hello World}
\begin{minted}[
	fontsize=\Large
	]{java}
// Exercise 01: Hello world
// Goal: Print a greeting message to the console.
public class Exercise01_Hello {
    public static void main(String[] args) {
        System.out.println("Hello, world!");
        System.out.println("This is Exercise 01.");
    }
}
\end{minted}
\end{frame}

\begin{frame}[fragile]{Como compilar e executar}
	\centering
	\includegraphics[height=6cm]{Como_compilar_e_executar_java.png}
\end{frame}

\begin{frame}{Desafio: experimente e descubra}
	\centering
	\includegraphics[height=7cm]{Desafio_mude_o_texto.png}
\end{frame}

\end{document}